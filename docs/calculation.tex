\documentclass{scrartcl}

\usepackage{listings}
\usepackage{amsmath}

\title{CosmoAI model}
\author{Hans Georg Schaathun}

\begin{document}
\maketitle

\begin{center}
\begin{tabular}{l|l|l}
\hline
  Class Variables & Mathematical Notation & Calculation \\
\hline
\hline
   \texttt{CHI}       & $\chi=\frac{chi_L}{\chi_S}$      & User Setting \\
   \texttt{size}      & Image size (not part of the model) & User Setting \\
   \texttt{einsteinR} & $R_E$                              & User Setting \\
   \texttt{sourceSize} & $\sigma$                          & User Setting \\
   \texttt{nterms}    & Number of terms after truncation    & User Setting \\
   \texttt{actualX}, \texttt{actualY} & $P_{\textrm{ACT}}$  & User Setting\\
   \texttt{apparentX}, \texttt{apparentY} 
         & $P_{\textrm{APP}}=\rho_1P_{\textrm{ACT}}$ & \texttt{updateXY()} \\
   \texttt{actualAbs} & $||P_{\textrm{ACT}}||$ & \texttt{updateXY()} \\
   \texttt{apparentAbs} & $||P_{\textrm{APP}}||$ & \texttt{updateXY()} \\
   \texttt{alphas\_val[m][s]},
   \texttt{betas\_val[m][s]}  & $\alpha,\beta$ \textbf{TODO}
        (floating point values)
        & \texttt{calculateAlphaBeta()} \\
   \texttt{alphas\_v[m][s]},
   \texttt{betas\_l[m][s]}  & $\alpha,\beta$ \textbf{TODO}
        (algebraic expressions)
        & \texttt{initAlphaBeta()} loading from \texttt{50.txt} \\
\hline
\end{tabular}
\end{center}

\begin{center}
\begin{tabular}{l|l|l}
\hline
   Intermediate Variables & Mathematical Notation & Calculation \\
\hline
\hline
   \texttt{xi1},\texttt{xi2}       & $\xi_1,\xi_2$     & Return value from \texttt{getDistortedPos} \\
   \texttt{ratio1},\texttt{ratio2} & $\rho_1,\rho_2$   & \texttt{updateXY}        \\
   \texttt{r}, \texttt{theta} & Polar coordinates $(r,\theta)$ &
      Arguments to \texttt{getDistortedPos} \\
   \texttt{c\_p} & $c_+=1+s/(m+1)$ & \texttt{getDistortedPos} \\
   \texttt{c\_m} & $c_+=1-s/(m+1)$ & \texttt{getDistortedPos} \\
\hline
\end{tabular}
\end{center}

\section{The \texttt{distort()} function}

\begin{lstlisting}
void Simulator::distort(int begin, int end, const cv::Mat& src, cv::Mat& dst) {
    // Iterate over the pixels in the image distorted image.
    // (row,col) are pixel co-ordinates
    for (int row = begin; row < end; row++) {
        for (int col = 0; col < dst.cols; col++) {

            int row_, col_;  // pixel co-ordinates in the apparent image
            std::pair<double, double> pos ;

            // Set coordinate system with origin at x=R
            double x = (col - apparentAbs - dst.cols / 2.0) * CHI;
            double y = (dst.rows / 2.0 - row) * CHI;

            // Calculate distance and angle of the point evaluated 
            // relative to center of lens (origin)
            double r = sqrt(x * x + y * y);
            double theta = atan2(y, x);

            pos = this->getDistortedPos(r, theta);

            // Translate to array index
            row_ = (int) round(src.rows / 2.0 - pos.second);
            col_ = (int) round(apparentAbs + src.cols / 2.0 + pos.first);

            // If (x', y') within source, copy value to imgDistorted
            if (row_ < src.rows && col_ < src.cols && row_ >= 0 && col_ >= 0) {
                auto val = src.at<uchar>(row_, col_);
                dst.at<uchar>(row, col) = val;
            }
        }
    }
}
\end{lstlisting}

Suppose the distorted image is an $m\times n$ matrix.
We rewrite the pixel coordinates $(i,j)$ as $(x,y)$ to get
a canonical Cartesian coordinate system centered at
the apparent location of the source.
\begin{align}
     x &= (j - ||P_{APP}|| - n/2)\cdot\chi \\
     y &= (-i + m/2)\cdot\chi 
\end{align}
Given the Cartesian coordinates $(x,y)$, we find Polar coordinates
$(r,\theta)$ as
\begin{align}
   r &= \sqrt{x^2+y^2}\\
   \theta &= 
     \begin{cases}
        \tan^{-1} \frac{y}{x}, \text{ if } x\ge0\\
        \pi + \tan^{-1} \frac{y}{x}, \text{ if } x<0
     \end{cases}
\end{align}

The \texttt{getDistortedPos} method implements the main coordinate distortion
functions and map $(r, \theta) \mapsto \xi = (\xi_1,\xi_2)$.

\begin{itemize}
   \item \textbf{TODO} There is a likely scaling error in this formula.
   \item \textbf{TODO} We should use $\eta$ for the position in the source plane and
        $\xi$ in the lens plane.  Then $\xi=\chi\eta$.
        This is not done consistently at the moment.
\end{itemize}

\end{document}
